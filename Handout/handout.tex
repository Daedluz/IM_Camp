\documentclass[12pt]{article}
\usepackage{settings}
\usepackage[utf8]{inputenc}

\title{\kemin}
\author{臺大資管營教學團隊\quad Michael}
\date{2021.12.5}

\begin{document}
\begin{CJK*}{UTF8}{bkai}
\maketitle
\setcounter{section}{-2}
\section{Compiler}
To save the time in our class and memory space in your computer, we will use an online compiler on Programiz. You may access it by clicking {\color{blue}\href{https://www.programiz.com/python-programming/online-compiler/}{this}} or scan the QR-code below.
\begin{center}
    \qrcode[height = 3cm]{https://www.programiz.com/python-programming/online-compiler/}
\end{center}
\section{Does it work?}
Try this line below on Programiz $\rightarrow$ main.py:
\begin{itemize}
    \item \texttt{print("Hello World!")}
\end{itemize}
Try this line below on Programiz $\rightarrow$ Shell:
\begin{itemize}
    \item \texttt{1 + 1}
\end{itemize}

\section{What is Python?}
Python is a high-level general-purpose programming language. It is created by Guido 
van Rossum in 1989 when he was looking for a project to keep him occupied in 
Christmas. It is now one of the most popular programming languages in the world.

\subsection{Why Python ?}
\begin{itemize}
    \item Works on different operating systems
    \item Simple syntax similar to English language
    \item Runs on interpreter systems, so that code can be executed as soon as it's written
    \item Hundreds of libraries
    \item Can be appied to various fields
\end{itemize}

\subsection{What can Python Do ?}
\begin{itemize}
    \item Handle Big data to perform statistic analysis
    \item Backend (server) of web applications
    \item AI and Machine Learning
    \item Programming applications
\end{itemize}

\section{Variables and \texttt{print()}}
\subsection{Variables}
There are many data types, one of which is integer, abbreviated \texttt{int}, in Python. We will focus on this data type today. Nevertheless, we still list some basic data types here for your further studying:
\begin{itemize}
    \item Integers
    \item Floating-Point Numbers
    \item Complex Numbers
    \item Strings
    \item Boolean Type
\end{itemize}
As a coder, we need ``variables'' to store some data for further use. For each variable assignment, it would be better to specify its data type. See the example below.
\subsection{\texttt{print()}}
\texttt{print()} is a function helping us to display the value of a variable.
\subsubsection{Example}
Try this on Programiz $\rightarrow$ main.py and see what happens.
\begin{lstlisting}[language = python]
    a = int(3)
    b = int(5)
    print(b)
    print(a)
\end{lstlisting}
\subsection{Formatted \texttt{print()}}
We can print a formatted string with \texttt{f''} or \texttt{f""}. \\
Inside this string, you can write a Python expression between \{ and \}
characters that can refer to variables or literal values. 
\subsubsection{Example}
\begin{lstlisting}[language =python]
    a = int(3)
    b = str('Michael')
    print(f'The value of a is {a}')
    print(f'my name is {b}')
\end{lstlisting}

Now, we believe you come familiar with the assignment operator \texttt{=} and a function \texttt{print()}. Wait... what do I mean by ``operator''?
\section{Arithmetic and Bitwise Operators}
We've learned how to store a value in a variable. However, we would like to play more tricks on those variables.
\subsection{Arithmetic Operators}
\begin{itemize}
    \item \texttt{+}
    \item \texttt{-}
    \item \texttt{*}
    \item \texttt{//}
    \item \texttt{**}
\end{itemize}
\subsection{Bitwise Operators}
Before going into this subsection, we'd better first know what binary representation is.
\setcounter{subsubsection}{-1}
\subsubsection*{Binary Representation}
In decimal representation, $7050$ actually means $ 7 \times 10^3 + 0 \times 10^2 + 5 \times 10^1 + 0 \times 10^0$.
\begin{itemize}
    \item Base: \underline{\phantom{xxxxxxxxx}}
    \item Digits: \underline{\phantom{xxxxxxxxx}}
\end{itemize}
What if the base is not ten?
\subsubsection*{Memos}
\vskip 200pt
\subsubsection*{Exercise}
\begin{itemize}
    \item What is the binary representation of 32 ?
    \vskip 150pt
    \item What is the binary representation of 102 ?
    \vskip 150pt
\end{itemize}
\newpage
\setcounter{subsubsection}{0}
\subsubsection{List of Bitwise Operators}
\begin{itemize}
    \item \texttt{\&} :
    Performs logical conjunction on the corresponding bits of its operands.
    \item \texttt{|} : 
    Performs logical disjunction. For each corresponding pair of bits. 
    \item \texttt{\^} :
    Performs exclusive disjunction on the bit pairs
    \item \texttt{>>} :
    Drop the n bits on the right. 
    \item \texttt{<<} :
    Moves the bits to the left by n bits. 
\end{itemize}
\subsection{Assignment Operators}
\begin{itemize}
    \item \texttt{=}
    \item \texttt{+=}
    \item $\vdots$
    \item \texttt{//=}
    \item \texttt{\&=}
    \item $\vdots$
    \item \texttt{<<=}
\end{itemize}
\section{Statements and Conditions}
Sometimes, we may want our program do different things based on different conditions.
\begin{itemize}
    \item \texttt{if}
    \item \texttt{else}
    \item \texttt{elif}
\end{itemize}
Moreover, things are usually complicated so we need some ``conjunctions''.
\begin{itemize}
    \item \texttt{and}
    \item \texttt{or}
\end{itemize}
\subsection{Exercise}
Write a program to determine whether a number is divisible by 7 or 5, between 1500 and 2700. If not, your program need to output whether the number is "Out of range" or "Not divisible".
\subsubsection{Hints}
\begin{enumerate}
    \item Store the number in a variable
    \item Write a if statement to check whether the number is in range
    \item Write a if statement inside the previous one and check if the number is divisible by 7 or 5
    \item print anything you want if both statement is true
    \item Figure out the rest by yourself !
\end{enumerate}
% Would nested if statement be too hard for them ?

% Lecturer, put an exercise you like here.
% \begin{enumerate}
%     \item {\color{blue} \href{https://atcoder.jp/contests/abc196/tasks/abc196_a}{Link}} and QR-code
%     \begin{center}
%         \qrcode[height = 3cm]{https://atcoder.jp/contests/abc196/tasks/abc196_a}
%     \end{center}
% \end{enumerate}
\section{Lists}
Lists are used to store multiple items in a single variable. \\
\subsection{Constructing a list}


\subsection{Accessing Elements in List}
We use "index" to access elements in a list. \\
the first item in lists has index 0, the second item has index 1 etc. \\
We can access the third element using \texttt{my\_list [2]}. \\
Use \texttt{[start:end]} to access a part of the list. \\ \\
{\large Examples} \\
L = [5, 10, 15, 20, 25, 30, 35, 40, 45, 50] \\
\texttt{L[2]} $\Rightarrow$ 15 \\
\texttt{L[-2]} $\Rightarrow$ 45 \\
\texttt{L[2:5]} $\Rightarrow$ [15, 20, 25] \\
\texttt{L[:3]} $\Rightarrow$ [5, 10, 15] \\
\texttt{L[6:]} $\Rightarrow$ [35, 40, 45, 50] \\

btw, you can print out the whole list by \texttt{print (my\_list)}
% \subsubsection{Small Exercise}
% Change the third number of \texttt{my\_list} to any number you want and print it.

\subsection{Add or Remove Elements}
\begin{itemize}
    \item Use \texttt{append()} to add element to the end of the list. \\
          e.g. \texttt{my\_list.append(50)} 
    \item Use \texttt{insert()} to add element to a specific index of the list. \\
          e.g. \texttt{my\_list.insert(i, elem)}
    \item Use \texttt{remove()} to remove an element in the list. \\
          e.g. \texttt{my\_list.remove(60)}
    \item Use \texttt{pop()} to remove an element in a specific index. \\
          e.g. \texttt{my\_list.pop(1)}
\end{itemize}

\subsection{Some Functions of Lists}
\begin{itemize}
    \item \texttt{len([5, 3, 1])} $\Rightarrow$ 3
    \item \texttt{max([1, 2, 3, 4, 5])} $\Rightarrow$ 5
    \item \texttt{min([0, 55, 3, 75])} $\Rightarrow$ 0
    \item \texttt{sum([1, 2, 3, 4, 5])} $\Rightarrow$ 15
\end{itemize}

\section{For Loop and While Loop}
\subsection{for loop}
We can use for loops to make our program do repetitive things \\
\begin{verbatim}
    for i in range (0, 5):
       (Do something)
\end{verbatim}
You could also use it to iterate through a list
\begin{verbatim}
    L = [5, 2, 88]
    for i in L :
       print (i)
\end{verbatim}
(Try on your computer to see what happens)
\subsection{while loop}
Execute a set of statements as long as a condition is true.\\
\begin{verbatim}
    i = 0
    while (i < 5):
       print (i)
       i += 1
\end{verbatim}
\subsection{\texttt{break()} and \texttt{continue}}
Use \texttt{break()} to break out of a loop. \\

Use \texttt{continue} to skip rest of the code and start a new iteration. \\

\subsection{Exercise}

\section{Functions}
A function is a block of code that runs when it's called.\\
You can pass parameters into a function, it'll do some calculation and return the value to you.\\
We've already used functions before ! (recall functions of lists)\\
It's great to use functions so that you don't have to copy and paste your code everywhere.
% Which is inconsistent and may lead to error

\begin{verbatim}
    def myFunction(num1, num2) :
        return num1**2 + num2**2
\end{verbatim}

\section{Conclusion}
Now that you've learned the basic syntax for Python, \\
you can explore various packages for Python ! \\
For example, numpy, pandas, matplotlib, scipy... \\\\
Have fun !

% beamer : https://www.overleaf.com/read/ycnhgdfcktnk

\end{CJK*}
\end{document}